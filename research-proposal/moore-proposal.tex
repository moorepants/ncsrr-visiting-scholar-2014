\documentclass[11pt]{article}

% enables Arial font
\usepackage{helvet}
\renewcommand{\familydefault}{\sfdefault}

\title{Identification of human locomotion control based on perturbed walking
and running data under the constraints of a biologically actutuated plant.}

\author{Jason K. Moore\\
  Postdoctoral Research Associate\\
  Parker Hannifin Human Motion \& Control Laboratory\\
  Cleveland State University
}

\begin{document}

\maketitle

\section{The goal}

I plan to extend the work presented in \cite{Wang2012} in such a way that the
controller parameters are identified by tracking experimental kinematic data in
the cost function as opposed to a task based costs. The key addition will be
the use of a data from walkers and runners under lateral and longitudinal
random perturbations. And other addtions will include:

\begin{itemize}
  \item Simple human walker/runner model developed in Opensim which is
    optimized for speed of simulation.
\end{itemize}

We believe that by utilizing data collected from walkers and runners while
being sufficiently and randomly perturbed and an indirect identification
approach that we can converge to more realistic controllers and control
parameters than predictive control methods have shown. Also, with careful
attention to simulation speed and optimization methods we can increase the
speed of the computations so that using time series on the order of 5 minutes
at 100 hZ (~250 steps) is computationally tractable.

\section{Introduction}

The current research focus at the Human Motion \& Control Laboratory in
Cleveland is focused on the identification and application of biolligically
inspired controllers to powered prothestics. My current role is based around
using data driven approaches to identify controllers capable of reproducing
natural human gait patterns from data collected from able bodied people in
walking and running. The current identified controllers are structured to map
to the sensors and acutuators available on non-nuerologically connected
prosthetics.

There are two approaches we are working with:

\begin{itemize}
  \item Direct identification given measurements of the controllers inputs and
    outputs.
  \item Indirect identifcation given the measuremetns of the system's
    kinematics and a known plant model.
\end{itemize}

The direct identification methods is attractive because it doesn't requier a
model of the plant. But it is limited due to the inability to acutally measure
the controller inputs and outputs directly and that it requires a suffciently
adequate noise model for the controller and along with persistent excitation to
reduce bias in the estimates of the model.

The indirect method has been becoming more attractive to us attractive because
any identification method can be applied with no effect from bias, but the
downside is that 

We are in need a sufficiently complex, but simple plant model that and
similarly a simple control model set that includes the ``true'' model. There
are many models that can be used, but a particularly interesting, sucessful.
and relevant platn model is the one presented by \cite{Wang2012}. Wang et. al,
demonstrated 

\section{How}

\subsection{Data}

We have a modern gait lab that includes a Motek Medical hardware and software
suite. The suite includes a Opsrey Digital RealTime Motion Capture system from
Motion Analysis and a R-Mill Treamdill from ForceLink. The treadmill is dual
belt with dual force plates and includes the ability to acutuate the treadmill
surface laterally and in pitch. The belts can accelerate up to $20 m/s^{-2}$,
the lateral motion of 0.1 m/s, and a pitching motion of 20 degrees/s. We have
developed a protocol that applies random longitudinal and lateral perturbations
over four minute trials sampled at 100 hz.

% TODO : Show example perturbed data.

We are currently collecting data from able-bodied subjects while walking and
running at speeds from 0.8 m/s to 4.0 m/s.

% TODO : Figure out what speeds we are actually using.

I will have rich sets of rich motion capture, ground reaction force, joint
torques, muscle force and length data available at the time of the visiting
scholarship from this system and from multiple able-bodied subjects.

\subsection{Plant Model}

There are a variety of plant models capable of walking and running available
for use, but my visit will focus on the development of a simple by sufficiently
complex plant model using Opensim that is capable of faster than real time
simulation times.

\subsubsection{Simplest Model}


\subsection{Controller Model}

\subsection{Identification}

\subsection{Contributions}

fast Simbody/Opensim based walking simualtion models
rich datasets of perturbed walking and running
framework for system identification of control mechanisms for walking

\section{My Background}

I'd say that I'm computationaly adept at modeling and simulation and have
experience with biomechanical systems and walking. I was initially trained at
UC Davis and specialized in Sports Biomechanics for my graduate work. I
co-developed and run a project called PyDy that provides an open source
modeling and simulation framework

Opensim and Simbody have not been a popular tool in the laboratories that I've
worked at and thus my experience with them is low. I've tended to develop my
own models with PyDy in the past. But I've spent time working on example
walking models

%Specific Aims: Review the rationale for your study, and state your hypotheses
%and specific aims.
%
%Methods: Provide an overview of the study design and methods. Clearly identify
%why an on-site collaboration would be beneficial to the proposed work.
%Potential bottlenecks should also be described with proposed solutions.
%
%Expected Results: State the expected outcome of your work and its impact on the
%field.
%
%Relevance to Rehabilitation: Articulate the importance of this project to
%rehabilitation.
%
%Contributions to the Biomechanical Simulation Community:  Describe the
%software, data, and/or models that will be made available to the biomechanics
%community at the end of the visit.
%
%Suitability of Applicant:  Explain why your background is appropriate for the
%proposed research problem. Describe your prior experience using OpenSim or
%other modeling and simulation approaches.

\end{document}
