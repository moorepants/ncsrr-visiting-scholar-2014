\documentclass[11pt]{article}

% enables Arial font
\usepackage{helvet}
\renewcommand{\familydefault}{\sfdefault}

\title{Identification of human locomotion control using on perturbed walking
and running data under the constraints of a biologically actutuated plant.}

\author{Jason K. Moore\\
  Postdoctoral Research Associate\\
  Parker Hannifin Human Motion \& Control Laboratory\\
  Cleveland State University
}

\begin{document}

\maketitle

\section{The Gist}

I plan to extend the optimal control work presented in \cite{Wang2012} in such
a way that the controller parameters are identified by tracking experimental
kinematic data in the cost function as opposed to a task based cost
optimization scheme. The key innovation that I will add will be the use of a
data from walkers and runners under lateral and longitudinal random
perturbations for the trajectory tracking. In this process, contributions will
be made to improve simulation accuracy and time along with speedups in
convergence to optimal solutions using the Opensim framework.

\section{The Hypothesis}

We believe that by utilizing data collected from walkers and runners while
being sufficiently and randomly perturbed using an indirect identification
approach with a sufficiently realistic plant model that we can converge to more
realistic controllers and control parameters than predictive control methods
have shown. Also, with careful attention to simulation speed and optimization
methods we can increase the speed of the computations so that using time series
on the order of 1 to 10 minutes at 100 hZ (~250 steps) is computationally
tractable using modern cloud computing

\section{Introduction}

The current research focus at the Human Motion \& Control Laboratory in
Cleveland is focused on the identification and application of biolligically
inspired controllers to powered prothestics. My current role is based around
using data driven approaches to identify controllers capable of reproducing
natural human gait patterns from data collected from able bodied people in
walking and running. The current identified controllers are structured to map
to the sensors and acutuators available on non-nuerologically connected
prosthetics.

There are two approaches we are working with:

\begin{itemize}
  \item Direct identification given measurements of the controllers inputs and
    outputs.
  \item Indirect identifcation given the measuremetns of the system's
    kinematics and a known plant model.
\end{itemize}

The direct identification methods is attractive because it doesn't requier a
model of the plant. But it is limited due to the inability to acutally measure
the controller inputs and outputs directly and that it requires a suffciently
adequate noise model for the controller and along with persistent excitation to
reduce bias in the estimates of the model.

The indirect method has been becoming more attractive to us attractive because
any identification method can be applied with no effect from bias, but the
downside is that 

We are in need a sufficiently complex, but simple plant model that and
similarly a simple control model set that includes the ``true'' model. There
are many models that can be used, but a particularly interesting, sucessful.
and relevant platn model is the one presented by \cite{Wang2012}. Wang et. al,
demonstrated 

\section{How}

\subsection{Data}

We have a modern gait lab that includes a Motek Medical hardware and software
suite. The suite includes a Opsrey Digital RealTime Motion Capture system from
Motion Analysis and a R-Mill Treamdill from ForceLink. The treadmill is dual
belt with dual force plates and includes the ability to acutuate the treadmill
surface laterally and in pitch. The belts can accelerate up to $20 m/s^{-2}$,
the lateral motion of 0.1 m/s, and a pitching motion of 20 degrees/s. We have
developed a protocol that applies random longitudinal and lateral perturbations
over four minute trials sampled at 100 hz.

% TODO : Show example perturbed data.

We are currently collecting data from able-bodied subjects while walking and
running at speeds from 0.8 m/s to 4.0 m/s.

% TODO : Figure out what speeds we are actually using.

I will have rich sets of rich motion capture, inertially compensated ground
reaction forces data along with estimates of joint kinematics, torques, muscle
force and length data available at the time of the visiting scholarship from
this system and from multiple able-bodied subjects.

\subsection{Identification}

\cite{Wang2012} approaches the ``identification'' of a plausible controller
from the predictive standpoint. A plant and controller model structure is
selected and the controller parameters are identified by miniminzing given a
cost function based only on minimizing metabolic effort and keeping several
tasks. I'd like to extend this concept by removing the task based costs and
replacing them with trajectory tracking of measured kinematics collected from
walkers and runners while being perturbed.

Wang has shown that his controller is able to produce visually motions that
also exhibit realistic joint torques, at least in walking. I hypothesize that
more realistic controller parameters can be identified by tracking kinematic
data from walkers and runners that are being perturbed with known (measured)
perturbations.

\begin{equation}
  J = \int_0^\infty \mathbf{x}\mathbf{bf}\mathbf{x}^T +
  \mathbf{u}\mathbf{R}\mathbf{u}^T
\end{equation}

This formulation will require simulating the system from 1 to 10 minutes of
real time and ensuring that the difference in the states and the experimental
measurements (sampled at 100 hz) a miminal. Wang's model simulated for 10s at
2400 Hz for accuracy resulting in 10 hour computations on a 50 core cluster to
converge on a solution. This means that using Wang's work as is, the
computations could take 60 to 600 hours for the trajectory tracking.

\subsection{Plant Model}

There are a variety of plant models capable of walking and running potentially
available for use, including Wang's Open Dynamics Engine based model, but my
visit will focus on the development of a simple but sufficiently complex plant
model based on lessons learned from Wang's implementation using Opensim that is
more accurate and capable of simuluating accurately much faster than real time.
\footnote{I've been in contact with the authors of \cite{Wang2012} and
understand that a Simbody based model may be close to opertaional at the time
of the visiting scholarship. I also asked Wang if the source code for the 2012
paper could be shared, but he seemed reluctant.}

Wang's model is a 3D 30 degree of freedom model which can be simulated at real
time on standard modern hardware. His plant includes simple contact for the
heel and ball of the feet and Hill type muscle models to actuate the joints
most important for gait analysis and simple joint torque specific

\begin{itemize}
  \item Simpler biomehcanical model, fewer degrees of freedom.
  \item Restrictions on motion of the subject during walking.
  \item Faster integration routines.
  \item Use explicit formualations of the system and direct collocation.
  \item Use initial guesses from Wang's results.
\end{itemize}

But small iterations are needed for
simulation with the explicit Euler method used in Open Dynamics Engine and the
shooting optimizations take 10 hours on a a 50 core cluster for optimziation of
10 seconds of simulation.

We'd like to simulate from 1 to 10 minutes of real time walking for trajectory
tracking.

Add specified position input.

\subsection{Controller Model}

I will start with Wang's controller model using his results as initial guesses
for the controller parameters.

\subsection{Contributions}

All of the products developed during, prior, and after the the visiting
scholarship that are related to this proposal will be made available under very
permissible licenses (BSD/MIT like and/or CC-BY) and shared via code and data
repositories on the world wide web. In paticular, these are the likely products
that will potentially be valuable to the community:

\begin{description}
  \item[Plant Models] The source code and any Opensim xml descriptions for all
    of the plant models will be released to the SimTK community and be hosted on
    Github.
  \item[Source Code] All of the source code will be shared via Github and
    structured such that all published results are completely reproducible.
  \item[Data] The anonimized human subject data will be shared via Figshare
    and/or instutional repositories and released under the CC-BY-4.0 License.
  \item[Papers] At least one journal paper will be published in an open access
    under the CC-BY-4.0 licenses along with supplementary material (source
    code and data) to reproduce the results. A preconfigured virutal machine
    that can run the computations on a cloud infastructure will accompany the
    results.
\end{description}

\section{My Background}

I'd say that I'm computationaly adept at modeling and simulation and have
experience with biomechanical systems and walking. I was initially trained at
UC Davis and specialized in Sports Biomechanics for my graduate work. I
co-develop and run a project called PyDy that provides an open source modeling
and simulation framework for complex multi-body systems.

Opensim and Simbody have not been a popular tool in the laboratories that I've
worked at and thus my experience with them is low. I've tended to develop my
own models with PyDy in the past. But I've spent time working on example
walking models

%Specific Aims: Review the rationale for your study, and state your hypotheses
%and specific aims.
%
%Methods: Provide an overview of the study design and methods. Clearly identify
%why an on-site collaboration would be beneficial to the proposed work.
%Potential bottlenecks should also be described with proposed solutions.
%
%Expected Results: State the expected outcome of your work and its impact on the
%field.
%
%Relevance to Rehabilitation: Articulate the importance of this project to
%rehabilitation.
%
%Contributions to the Biomechanical Simulation Community:  Describe the
%software, data, and/or models that will be made available to the biomechanics
%community at the end of the visit.
%
%Suitability of Applicant:  Explain why your background is appropriate for the
%proposed research problem. Describe your prior experience using OpenSim or
%other modeling and simulation approaches.

\end{document}
